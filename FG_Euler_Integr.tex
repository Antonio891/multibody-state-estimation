\documentclass[12pt,a4paper]{book}
\usepackage[latin1]{inputenc}
\usepackage[english]{babel}
\usepackage{amsmath}
\usepackage{amsfonts}
\usepackage{amssymb}
\usepackage{graphicx}
\usepackage{fourier}
\usepackage[left=2cm,right=2cm,top=2cm,bottom=2cm]{geometry}
\author{Antonio Leanza}
\title{REPORT on Factor Graph - First order integrators}

\begin{document}
\maketitle

\section*{About the document}
This document is only a report about \textit{First order Integrators} implementation by means of \textit{Factor Graph} method. These pages aim to be a base on which build a deeper discussion on this argument, by adding and/or changing what will be necessary.

\section*{Euler integration}
\textit{Euler Method} is one of the simplest methods for numerical solution of the first order Ordinary Differential Equations (ODE). For dynamic systems, the indipendent variable is usally the \textit{time} $t$, thus the first order ODE will be
\begin{equation}
\dot{\mathbf{x}}(t)=f\left(\mathbf{x}(t),\mathbf{u}(t),t\right)+\mathbf{w}(t)
\label{1ODE}
\end{equation}
where $\dot{\mathbf{x}}(t)$ is the derivative of $\mathbf{x}(t)$ respect to $t$, $\mathbf{u}(t)$ is the eventual exogenous input and $\mathbf{w}$ is the uncertainty associated to the system. It is convenient to discretize the above equation as below, with $k \in \mathbb{N}|1 \le k \le K$
\begin{equation}
\mathbf{x}_{k+1}=f\left(\mathbf{x}_k,\mathbf{u}_k\right)+\mathbf{w}_k
\label{1ODEbis}
\end{equation}
The \textit{Euler Integrator} is grounded on the equation (\ref{1ODEbis}) by means of the following formula
\begin{equation}
\mathbf{x}_{k+1}=\mathbf{x}_k+\Delta t\cdot \mathbf{v}_k+\mathbf{w}_k
\label{EuIn0}
\end{equation}
where the meaning of $\mathbf{w}_k$ is the error due to the numerical approximation of the exact solution and $\Delta t$ is the integration step size. Furthermore, $\mathbf{v}_k=\dot{\mathbf{x}}_k$.
\\For sake of simpleness, (\ref{EuIn0}) will be written as below
 \begin{equation}
\mathbf{x}_2=\mathbf{x}_1+\Delta t\cdot \mathbf{v}_1+\mathbf{w}_1
\label{EuIn1}
\end{equation}
It is worth to underline that Euler Method is already a \textit{linear} integration method, therefore it does not need any further linearizations, indeed, as we will see, the Jacobian matrix is constant.

\subsection*{Euler integrator factor}
Fig. (\ref{FG}) shows the \textit{Factor Graph} modelling the \textit{Euler Integrator}. In this graph the black node indicate the \textit{factor} that binds all variables (white nodes) by means of their density probability function that in turn will be maximized.
\begin{figure}[hbtp]
\centering
\includegraphics[scale=0.75]{EulerFig/FG.jpg}
\caption{Factor Graph of the Euler Integrator}
\label{FG}
\end{figure}

The function associated to the factor in Fig. (\ref{FG}) is the following
\begin{equation}
f\left(\mathbf{x}_1,\mathbf{v}_1,\mathbf{x}_2\right)=P\left(\mathbf{x}_1\right)P\left(\mathbf{x}_2\right)P\left(\mathbf{x}_2|\mathbf{x}_1,\mathbf{v}_1\right)
\label{fFG}
\end{equation}
where $P(\cdot )$ is the density probability function to maximize
\begin{equation}
\max_{\mathbf{x}} \hspace*{0.3cm} f\left(\mathbf{x}_1,\mathbf{v}_1,\mathbf{x}_2\right)
\end{equation}
with $\mathbf{x}$ a vector containing the three involved variables. 
\\The aim is to minimize the error between the actual $\mathbf{x}_2$ and its prediction by using Euler Method, \textit{alias}
\begin{equation}
\text{e}\left(\mathbf{x}_1,\mathbf{v}_1,\mathbf{x}_2\right)= \mathbf{x}_2 - \mathbf{x}_1 - \Delta t \cdot \mathbf{v}_1
\label{factor}
\end{equation}
namely the function into the factor (black node in Fig. (\ref{FG})). The error function $\text{e}\left(\mathbf{x}_1,\mathbf{v}_1,\mathbf{x}_2\right)$ of equation (\ref{factor}) has a different meaning respect to the error $\mathbf{w}_1$ of equation (\ref{EuIn1}), because it is associated to an estimation instead to the model inaccuracies due to the truncation. 
\\This factor is achieved by coming from both sides of the graph, hence we need some prior knowledge about the involved variables, that is their distributions. A gaussian distribution is assumed, centered on the variables themselves with own covariance matrices:
\begin{equation}
\mathbf{x}_1 \sim \mathcal{N}\left(\mathbf{x}_1,\mathbf{C}_{\mathbf{x}_1}\right) \hspace*{0.5cm}
\mathbf{v}_1 \sim \mathcal{N}\left(\mathbf{v}_1,\mathbf{C}_{\mathbf{v}_1}\right) \hspace*{0.5cm}
\mathbf{x}_2 \sim \mathcal{N}\left(\mathbf{x}_2,\mathbf{C}_{\mathbf{x}_2}\right)
\label{VarStat}
\end{equation}
The entity of the various covariance matrices depends by the trusting to the knowledge of the variables. To compute the covariance of the error, the following equation is exploited
\begin{equation}
\mathbf{C}_{\text{e}} = \nabla_\mathbf{x}\text{e} \cdot \mathbf{C} \cdot \left[\nabla_\mathbf{x}\text{e}\right]^T
\label{UncertPropagation}
\end{equation}
where
\begin{equation*}
\mathbf{x} = 
\begin{bmatrix}
\mathbf{x}_1 \\ \mathbf{v}_1 \\ \mathbf{x}_2 
\end{bmatrix}, 
\hspace*{0.4cm}
\mathbf{C}=\text{diag}\left(\mathbf{C}_{\mathbf{x}_1},\mathbf{C}_{\mathbf{v}_1},\mathbf{C}_{\mathbf{x}_2}\right),
\hspace*{0.4cm}
\nabla_\mathbf{x}\text{e} = 
\left[-\mathbf{I}_N \hspace*{0.2cm} -\Delta t \cdot \mathbf{I}_N \hspace*{0.3cm} \mathbf{I}_N\right]
\end{equation*}
that is the rule of the uncertainty propagation for non linear systems, with $\mathbf{I}_N \in \mathbb{R}^{N\text{x}N}$ the identity matrix and $N$ the number of elements of each variable. When a system is linear, as the case at hand, Jacobian matrix becomes the matrix of equation parameters. Therefore, the covariance matrix of the error becomes
\begin{equation}
\mathbf{C}_{\text{e}}=\mathbf{C}_{\mathbf{x}_1}+\Delta t^2 \mathbf{C}_{\mathbf{v}_1}+\mathbf{C}_{\mathbf{x}_2}
\label{errCov}
\end{equation}
Remebering that the target is to minimize the error function $\text{e}(\mathbf{x})$, it is sufficient to solve
\begin{equation}
\text{arg}\min_\mathbf{x}\left[\text{e}^T\mathbf{C}_\text{e}^{-1}\text{e}\right]
\label{LS}
\end{equation}

\subsection*{Implementation}
In order to evaluate $\mathbf{x}_2$, the variables $\mathbf{x}_1$ and $\mathbf{v}_1$ must to be known \textit{a priori}, for this reason two \textit{Prioir Factors} are added to the graph before these variables (Fig. (\ref{EulerIntFG})). Their discussion is beyond this document. The only parameter inside the \textit{Euler integrator factor} is $\Delta t$ and $\nabla_\mathbf{x}\text{e}$ is implemented into the factor to evolve the state both for linear and nonlinear cases.\\
\begin{figure}[hbtp]
\centering
\includegraphics[scale=0.85]{EulerFig/EulerIntFG_fig.jpg}
\caption{Factor Graph of the Euler Integrator}
\label{EulerIntFG}
\end{figure}
Below a simple application of the Euler method is shown, by appling \textit{Euler Integrator Factor} two times ($\mathbf{e1}$, $\mathbf{e2}$). As one can see from Fig. (\ref{EuIntExFG}), three \textit{Prior Factors} ($\mathbf{p1}$, $\mathbf{p2}$ and $\mathbf{p3}$) are exploited to evaluate prior variable ($\mathbf{x}_1$, $\mathbf{v}_1$ and $\mathbf{v}_2$ respectively). $\mathbf{e1}$ evaluates $\mathbf{x}_2$ by using $\mathbf{x}_1$ and $\mathbf{v}_1$ and $\mathbf{e2}$ evaluates in turn $\mathbf{x}_3$ by using $\mathbf{x}_2$ and $\mathbf{v}_2$; both $\mathbf{e1}$ and $\mathbf{e2}$ use an internal step size $\Delta t=1$ seconds.\\
\begin{figure}[hbtp]
\centering
\includegraphics[scale=0.75]{EulerFig/EuIntEx_fig.jpg}
\caption{Factor Graph of the Euler Integrator}
\label{EuIntExFG}
\end{figure}
For a better on screen visualization, each name variable in associated to the variable \textbf{Key} inside the factor constructor, as shown in Tab. (\ref{KeyTab}).\\
\begin{table}[]
\centering
\begin{tabular}{c|c}
\textbf{Name} & \textbf{Key} \\ \hline
$\mathbf{x}_1$         & 1            \\
$\mathbf{x}_2$         & 2            \\
$\mathbf{x}_3$         & 3            \\
$\mathbf{v}_1$         & 4            \\
$\mathbf{v}_2$         & 5           
\end{tabular}
\caption{Variable name and relative Key parameter inside the factor}
\label{KeyTab}
\end{table}
If we have the following instances
\begin{equation*}
\mathbf{x}_1= \begin{bmatrix}
0 \\ 0
\end{bmatrix} \hspace*{0.3cm}
\mathbf{v}_1= \begin{bmatrix}
1 \\ 2
\end{bmatrix} \hspace*{0.3cm}
\mathbf{v}_2= \begin{bmatrix}
2 \\ 3
\end{bmatrix} \hspace*{0.3cm}
\end{equation*}
the condition above give
\begin{eqnarray*}
\mathbf{x}_2=\mathbf{x}_1+\Delta t \cdot \mathbf{v}_1 =
\begin{bmatrix}
1 \\ 2
\end{bmatrix} \\
\mathbf{x}_3=\mathbf{x}_2+\Delta t \cdot \mathbf{v}_2 =
\begin{bmatrix}
3 \\ 5
\end{bmatrix}
\end{eqnarray*}

\section*{Trapezoidal integration method to solve ODEs}
Another simple method to solve differential equations can be obtain by recurring to the \textit{Trapezium rule}, that is
\begin{equation}
\mathbf{x}_{k+1}=\mathbf{x}_k+\frac{\Delta t}{2}\left( \mathbf{v}_k+\mathbf{v}_{k+1}\right) +\mathbf{w}_k
\label{TrapInt0}
\end{equation}
The equation (\ref{TrapInt0}) leads to the next one
\begin{equation}
\text{e}\left(\mathbf{x}_1,\mathbf{v}_1,\mathbf{x}_2,\mathbf{v}_2\right)= \mathbf{x}_2 - \mathbf{x}_1 - \frac{\Delta t}{2} \cdot \mathbf{v}_1 - \frac{\Delta t}{2} \cdot \mathbf{v}_2
\label{TrapInt1}
\end{equation}
This method is quite similar to the previous, indeed it needs only one parameter (still $\Delta t$) but, unlike the Euler Method, this one involves four variables, so it is a little more articulated. Fig. (\ref{TrapIntFG}) shows the factor graph of the \textit{Trapezoidal Integrator}, that links the four variables $\mathbf{x}_1$, $\mathbf{x}_2$, $\mathbf{v}_1$ and $\mathbf{v}_2$, three of which need an own Prior factor.
\\By collecting the variables into the vector 
\begin{equation*}
\mathbf{x}=
\begin{bmatrix}
\mathbf{x}_1 \\ \mathbf{v}_1 \\ \mathbf{x}_2 \\ \mathbf{v}_2
\end{bmatrix}
\end{equation*}
the Jacobian inside the \textit{Trapezoidal Integrator Factor} will be
\begin{equation}
\nabla_\mathbf{x}\text{e}= 
\left[-\mathbf{I}_N \hspace*{0.2cm} -\frac{\Delta t}{2} \cdot \mathbf{I}_N \hspace*{0.3cm} \mathbf{I}_N \hspace*{0.2cm} -\frac{\Delta t}{2} \cdot \mathbf{I}_N\right]
\end{equation}

\begin{figure}[hbtp]
\centering
\includegraphics[scale=0.85]{EulerFig/TrapIntFG_fig.jpg}
\caption{Factor Graph of the Trapezoidal Integrator}
\label{TrapIntFG}
\end{figure}
The remainder follows as the previous integrator factor.



\end{document}